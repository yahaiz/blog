% Options for packages loaded elsewhere
% Options for packages loaded elsewhere
\PassOptionsToPackage{unicode}{hyperref}
\PassOptionsToPackage{hyphens}{url}
\PassOptionsToPackage{dvipsnames,svgnames,x11names}{xcolor}
%
\documentclass[
  letterpaper,
  DIV=11,
  numbers=noendperiod]{scrartcl}
\usepackage{xcolor}
\usepackage{amsmath,amssymb}
\setcounter{secnumdepth}{-\maxdimen} % remove section numbering
\usepackage{iftex}
\ifPDFTeX
  \usepackage[T1]{fontenc}
  \usepackage[utf8]{inputenc}
  \usepackage{textcomp} % provide euro and other symbols
\else % if luatex or xetex
  \usepackage{unicode-math} % this also loads fontspec
  \defaultfontfeatures{Scale=MatchLowercase}
  \defaultfontfeatures[\rmfamily]{Ligatures=TeX,Scale=1}
\fi
\usepackage{lmodern}
\ifPDFTeX\else
  % xetex/luatex font selection
\fi
% Use upquote if available, for straight quotes in verbatim environments
\IfFileExists{upquote.sty}{\usepackage{upquote}}{}
\IfFileExists{microtype.sty}{% use microtype if available
  \usepackage[]{microtype}
  \UseMicrotypeSet[protrusion]{basicmath} % disable protrusion for tt fonts
}{}
\makeatletter
\@ifundefined{KOMAClassName}{% if non-KOMA class
  \IfFileExists{parskip.sty}{%
    \usepackage{parskip}
  }{% else
    \setlength{\parindent}{0pt}
    \setlength{\parskip}{6pt plus 2pt minus 1pt}}
}{% if KOMA class
  \KOMAoptions{parskip=half}}
\makeatother
% Make \paragraph and \subparagraph free-standing
\makeatletter
\ifx\paragraph\undefined\else
  \let\oldparagraph\paragraph
  \renewcommand{\paragraph}{
    \@ifstar
      \xxxParagraphStar
      \xxxParagraphNoStar
  }
  \newcommand{\xxxParagraphStar}[1]{\oldparagraph*{#1}\mbox{}}
  \newcommand{\xxxParagraphNoStar}[1]{\oldparagraph{#1}\mbox{}}
\fi
\ifx\subparagraph\undefined\else
  \let\oldsubparagraph\subparagraph
  \renewcommand{\subparagraph}{
    \@ifstar
      \xxxSubParagraphStar
      \xxxSubParagraphNoStar
  }
  \newcommand{\xxxSubParagraphStar}[1]{\oldsubparagraph*{#1}\mbox{}}
  \newcommand{\xxxSubParagraphNoStar}[1]{\oldsubparagraph{#1}\mbox{}}
\fi
\makeatother

\usepackage{color}
\usepackage{fancyvrb}
\newcommand{\VerbBar}{|}
\newcommand{\VERB}{\Verb[commandchars=\\\{\}]}
\DefineVerbatimEnvironment{Highlighting}{Verbatim}{commandchars=\\\{\}}
% Add ',fontsize=\small' for more characters per line
\usepackage{framed}
\definecolor{shadecolor}{RGB}{241,243,245}
\newenvironment{Shaded}{\begin{snugshade}}{\end{snugshade}}
\newcommand{\AlertTok}[1]{\textcolor[rgb]{0.68,0.00,0.00}{#1}}
\newcommand{\AnnotationTok}[1]{\textcolor[rgb]{0.37,0.37,0.37}{#1}}
\newcommand{\AttributeTok}[1]{\textcolor[rgb]{0.40,0.45,0.13}{#1}}
\newcommand{\BaseNTok}[1]{\textcolor[rgb]{0.68,0.00,0.00}{#1}}
\newcommand{\BuiltInTok}[1]{\textcolor[rgb]{0.00,0.23,0.31}{#1}}
\newcommand{\CharTok}[1]{\textcolor[rgb]{0.13,0.47,0.30}{#1}}
\newcommand{\CommentTok}[1]{\textcolor[rgb]{0.37,0.37,0.37}{#1}}
\newcommand{\CommentVarTok}[1]{\textcolor[rgb]{0.37,0.37,0.37}{\textit{#1}}}
\newcommand{\ConstantTok}[1]{\textcolor[rgb]{0.56,0.35,0.01}{#1}}
\newcommand{\ControlFlowTok}[1]{\textcolor[rgb]{0.00,0.23,0.31}{\textbf{#1}}}
\newcommand{\DataTypeTok}[1]{\textcolor[rgb]{0.68,0.00,0.00}{#1}}
\newcommand{\DecValTok}[1]{\textcolor[rgb]{0.68,0.00,0.00}{#1}}
\newcommand{\DocumentationTok}[1]{\textcolor[rgb]{0.37,0.37,0.37}{\textit{#1}}}
\newcommand{\ErrorTok}[1]{\textcolor[rgb]{0.68,0.00,0.00}{#1}}
\newcommand{\ExtensionTok}[1]{\textcolor[rgb]{0.00,0.23,0.31}{#1}}
\newcommand{\FloatTok}[1]{\textcolor[rgb]{0.68,0.00,0.00}{#1}}
\newcommand{\FunctionTok}[1]{\textcolor[rgb]{0.28,0.35,0.67}{#1}}
\newcommand{\ImportTok}[1]{\textcolor[rgb]{0.00,0.46,0.62}{#1}}
\newcommand{\InformationTok}[1]{\textcolor[rgb]{0.37,0.37,0.37}{#1}}
\newcommand{\KeywordTok}[1]{\textcolor[rgb]{0.00,0.23,0.31}{\textbf{#1}}}
\newcommand{\NormalTok}[1]{\textcolor[rgb]{0.00,0.23,0.31}{#1}}
\newcommand{\OperatorTok}[1]{\textcolor[rgb]{0.37,0.37,0.37}{#1}}
\newcommand{\OtherTok}[1]{\textcolor[rgb]{0.00,0.23,0.31}{#1}}
\newcommand{\PreprocessorTok}[1]{\textcolor[rgb]{0.68,0.00,0.00}{#1}}
\newcommand{\RegionMarkerTok}[1]{\textcolor[rgb]{0.00,0.23,0.31}{#1}}
\newcommand{\SpecialCharTok}[1]{\textcolor[rgb]{0.37,0.37,0.37}{#1}}
\newcommand{\SpecialStringTok}[1]{\textcolor[rgb]{0.13,0.47,0.30}{#1}}
\newcommand{\StringTok}[1]{\textcolor[rgb]{0.13,0.47,0.30}{#1}}
\newcommand{\VariableTok}[1]{\textcolor[rgb]{0.07,0.07,0.07}{#1}}
\newcommand{\VerbatimStringTok}[1]{\textcolor[rgb]{0.13,0.47,0.30}{#1}}
\newcommand{\WarningTok}[1]{\textcolor[rgb]{0.37,0.37,0.37}{\textit{#1}}}

\usepackage{longtable,booktabs,array}
\usepackage{calc} % for calculating minipage widths
% Correct order of tables after \paragraph or \subparagraph
\usepackage{etoolbox}
\makeatletter
\patchcmd\longtable{\par}{\if@noskipsec\mbox{}\fi\par}{}{}
\makeatother
% Allow footnotes in longtable head/foot
\IfFileExists{footnotehyper.sty}{\usepackage{footnotehyper}}{\usepackage{footnote}}
\makesavenoteenv{longtable}
\usepackage{graphicx}
\makeatletter
\newsavebox\pandoc@box
\newcommand*\pandocbounded[1]{% scales image to fit in text height/width
  \sbox\pandoc@box{#1}%
  \Gscale@div\@tempa{\textheight}{\dimexpr\ht\pandoc@box+\dp\pandoc@box\relax}%
  \Gscale@div\@tempb{\linewidth}{\wd\pandoc@box}%
  \ifdim\@tempb\p@<\@tempa\p@\let\@tempa\@tempb\fi% select the smaller of both
  \ifdim\@tempa\p@<\p@\scalebox{\@tempa}{\usebox\pandoc@box}%
  \else\usebox{\pandoc@box}%
  \fi%
}
% Set default figure placement to htbp
\def\fps@figure{htbp}
\makeatother





\setlength{\emergencystretch}{3em} % prevent overfull lines

\providecommand{\tightlist}{%
  \setlength{\itemsep}{0pt}\setlength{\parskip}{0pt}}



 


\KOMAoption{captions}{tableheading}
\makeatletter
\@ifpackageloaded{tcolorbox}{}{\usepackage[skins,breakable]{tcolorbox}}
\@ifpackageloaded{fontawesome5}{}{\usepackage{fontawesome5}}
\definecolor{quarto-callout-color}{HTML}{909090}
\definecolor{quarto-callout-note-color}{HTML}{0758E5}
\definecolor{quarto-callout-important-color}{HTML}{CC1914}
\definecolor{quarto-callout-warning-color}{HTML}{EB9113}
\definecolor{quarto-callout-tip-color}{HTML}{00A047}
\definecolor{quarto-callout-caution-color}{HTML}{FC5300}
\definecolor{quarto-callout-color-frame}{HTML}{acacac}
\definecolor{quarto-callout-note-color-frame}{HTML}{4582ec}
\definecolor{quarto-callout-important-color-frame}{HTML}{d9534f}
\definecolor{quarto-callout-warning-color-frame}{HTML}{f0ad4e}
\definecolor{quarto-callout-tip-color-frame}{HTML}{02b875}
\definecolor{quarto-callout-caution-color-frame}{HTML}{fd7e14}
\makeatother
\makeatletter
\@ifpackageloaded{caption}{}{\usepackage{caption}}
\AtBeginDocument{%
\ifdefined\contentsname
  \renewcommand*\contentsname{Table of contents}
\else
  \newcommand\contentsname{Table of contents}
\fi
\ifdefined\listfigurename
  \renewcommand*\listfigurename{List of Figures}
\else
  \newcommand\listfigurename{List of Figures}
\fi
\ifdefined\listtablename
  \renewcommand*\listtablename{List of Tables}
\else
  \newcommand\listtablename{List of Tables}
\fi
\ifdefined\figurename
  \renewcommand*\figurename{Figure}
\else
  \newcommand\figurename{Figure}
\fi
\ifdefined\tablename
  \renewcommand*\tablename{Table}
\else
  \newcommand\tablename{Table}
\fi
}
\@ifpackageloaded{float}{}{\usepackage{float}}
\floatstyle{ruled}
\@ifundefined{c@chapter}{\newfloat{codelisting}{h}{lop}}{\newfloat{codelisting}{h}{lop}[chapter]}
\floatname{codelisting}{Listing}
\newcommand*\listoflistings{\listof{codelisting}{List of Listings}}
\makeatother
\makeatletter
\makeatother
\makeatletter
\@ifpackageloaded{caption}{}{\usepackage{caption}}
\@ifpackageloaded{subcaption}{}{\usepackage{subcaption}}
\makeatother
\usepackage{bookmark}
\IfFileExists{xurl.sty}{\usepackage{xurl}}{} % add URL line breaks if available
\urlstyle{same}
\hypersetup{
  pdftitle={Java Scanner: The Ultimate Beginner's Guide},
  pdfauthor={Yahya Izadi; Pooyan Motamedi},
  colorlinks=true,
  linkcolor={blue},
  filecolor={Maroon},
  citecolor={Blue},
  urlcolor={Blue},
  pdfcreator={LaTeX via pandoc}}


\title{Java Scanner: The Ultimate Beginner's Guide}
\usepackage{etoolbox}
\makeatletter
\providecommand{\subtitle}[1]{% add subtitle to \maketitle
  \apptocmd{\@title}{\par {\large #1 \par}}{}{}
}
\makeatother
\subtitle{How to make your Java programs listen to the user}
\author{Yahya Izadi \and Pooyan Motamedi}
\date{2025-11-24}
\begin{document}
\maketitle


\subsection{Introduction: The Scanner
Machine}\label{introduction-the-scanner-machine}

By default, the variables you write in code are fixed. But what if we
want to ask the user for the numbers?

Think of the \textbf{Scanner} like a physical scanner machine in an
office. Usually, when you put a paper in a physical scanner, it scans
the \textbf{entire} page at once.

In Java, our Scanner is a bit different. It looks at what the user types
on the keyboard, but we can be very specific about \textbf{what} to
scan. We can tell it: \emph{``Only scan the next number (int)''} or
\emph{``Only scan the next decimal point (float).''}

\begin{center}\rule{0.5\linewidth}{0.5pt}\end{center}

\subsection{The 3 Steps}\label{the-3-steps}

To use the Scanner, follow these three steps.

\subsubsection{Step 1: The Import Line}\label{step-1-the-import-line}

Java hides the Scanner tool in a specific library. You need to tell Java
where to find it. Put this line at the \textbf{very top} of your file.

\begin{Shaded}
\begin{Highlighting}[]
\KeywordTok{import} \ImportTok{java}\OperatorTok{.}\ImportTok{util}\OperatorTok{.}\ImportTok{Scanner}\OperatorTok{;}
\end{Highlighting}
\end{Shaded}

\subsubsection{Step 2: Turn on the
Scanner}\label{step-2-turn-on-the-scanner}

Inside your code (in the \texttt{main} section), you need to write a
specific line to set up the scanner.

\begin{tcolorbox}[enhanced jigsaw, leftrule=.75mm, opacitybacktitle=0.6, bottomtitle=1mm, toptitle=1mm, breakable, arc=.35mm, colback=white, opacityback=0, colframe=quarto-callout-note-color-frame, bottomrule=.15mm, colbacktitle=quarto-callout-note-color!10!white, rightrule=.15mm, coltitle=black, left=2mm, title=\textcolor{quarto-callout-note-color}{\faInfo}\hspace{0.5em}{Note}, titlerule=0mm, toprule=.15mm]

This line looks a bit complicated with keywords like \texttt{new} and
\texttt{System.in}. \textbf{For now, please just accept this line as a
rule.} Copy and paste it exactly as it is.

\end{tcolorbox}

\begin{Shaded}
\begin{Highlighting}[]
\CommentTok{// This line prepares the scanner to read from the keyboard}
\BuiltInTok{Scanner}\NormalTok{ input }\OperatorTok{=} \KeywordTok{new} \BuiltInTok{Scanner}\OperatorTok{(}\BuiltInTok{System}\OperatorTok{.}\FunctionTok{in}\OperatorTok{);}
\end{Highlighting}
\end{Shaded}

\subsubsection{Step 3: Scan the Data}\label{step-3-scan-the-data}

Now, you command the scanner to wait for the user to type something and
store it in a variable.

\begin{Shaded}
\begin{Highlighting}[]
\DataTypeTok{int}\NormalTok{ age }\OperatorTok{=}\NormalTok{ input}\OperatorTok{.}\FunctionTok{nextInt}\OperatorTok{();}
\end{Highlighting}
\end{Shaded}

\begin{center}\rule{0.5\linewidth}{0.5pt}\end{center}

\subsection{Cheat Sheet: Which Command to
Use?}\label{cheat-sheet-which-command-to-use}

Depending on the \textbf{Data Type} (variable) you want to fill, you
must use a specific command.

\begin{longtable}[]{@{}lll@{}}
\toprule\noalign{}
Data Type & Command & Example Use Case \\
\midrule\noalign{}
\endhead
\bottomrule\noalign{}
\endlastfoot
\texttt{int} & \texttt{input.nextInt()} & Whole numbers (Age, Year) \\
\texttt{long} & \texttt{input.nextLong()} & Large whole numbers
(Population) \\
\texttt{float} & \texttt{input.nextFloat()} & Decimal numbers
(Temperature) \\
\texttt{double} & \texttt{input.nextDouble()} & Decimal numbers (Price,
GPA) \\
\texttt{boolean} & \texttt{input.nextBoolean()} & True/False answers \\
\end{longtable}

\begin{center}\rule{0.5\linewidth}{0.5pt}\end{center}

\subsection{💻 Full Code Example}\label{full-code-example}

Here is a complete program. It asks for a student's number and age.

\begin{Shaded}
\begin{Highlighting}[]
\KeywordTok{import} \ImportTok{java}\OperatorTok{.}\ImportTok{util}\OperatorTok{.}\ImportTok{Scanner}\OperatorTok{;} \CommentTok{// Step 1}

\KeywordTok{public} \KeywordTok{class}\NormalTok{ Main }\OperatorTok{\{}
    \KeywordTok{public} \DataTypeTok{static} \DataTypeTok{void} \FunctionTok{main}\OperatorTok{(}\BuiltInTok{String}\OperatorTok{[]}\NormalTok{ args}\OperatorTok{)} \OperatorTok{\{}
    
        \CommentTok{// Step 2: Setup the Scanner }
        \CommentTok{// (Remember: Just copy this line for now!)}
        \BuiltInTok{Scanner}\NormalTok{ input }\OperatorTok{=} \KeywordTok{new} \BuiltInTok{Scanner}\OperatorTok{(}\BuiltInTok{System}\OperatorTok{.}\FunctionTok{in}\OperatorTok{);}

        \CommentTok{// Step 3: Ask and Scan}
        
        \CommentTok{// 1. Ask for an Integer}
        \BuiltInTok{System}\OperatorTok{.}\FunctionTok{out}\OperatorTok{.}\FunctionTok{print}\OperatorTok{(}\StringTok{"Enter your Student Number: "}\OperatorTok{);}
        \DataTypeTok{int}\NormalTok{ studentNumber }\OperatorTok{=}\NormalTok{ input}\OperatorTok{.}\FunctionTok{nextInt}\OperatorTok{();}

        \CommentTok{// 2. Ask for an Integer}
        \BuiltInTok{System}\OperatorTok{.}\FunctionTok{out}\OperatorTok{.}\FunctionTok{print}\OperatorTok{(}\StringTok{"Enter your age: "}\OperatorTok{);}
        \DataTypeTok{int}\NormalTok{ age }\OperatorTok{=}\NormalTok{ input}\OperatorTok{.}\FunctionTok{nextInt}\OperatorTok{();}

        \CommentTok{// Print the results back to see if it worked}
        \BuiltInTok{System}\OperatorTok{.}\FunctionTok{out}\OperatorTok{.}\FunctionTok{println}\OperatorTok{(}\StringTok{"{-}{-}{-} Result {-}{-}{-}"}\OperatorTok{);}
        \BuiltInTok{System}\OperatorTok{.}\FunctionTok{out}\OperatorTok{.}\FunctionTok{println}\OperatorTok{(}\StringTok{"Student Number: "} \OperatorTok{+}\NormalTok{ studentNumber}\OperatorTok{);}
        \BuiltInTok{System}\OperatorTok{.}\FunctionTok{out}\OperatorTok{.}\FunctionTok{println}\OperatorTok{(}\StringTok{"Age: "} \OperatorTok{+}\NormalTok{ age}\OperatorTok{);}
    \OperatorTok{\}}
\OperatorTok{\}}
\end{Highlighting}
\end{Shaded}

\begin{center}\rule{0.5\linewidth}{0.5pt}\end{center}

\subsection{Summary}\label{summary}

\begin{enumerate}
\def\labelenumi{\arabic{enumi}.}
\tightlist
\item
  \textbf{Import} the tool at the top:
  \texttt{import\ java.util.Scanner;}
\item
  \textbf{Setup} the tool inside main:
  \texttt{Scanner\ input\ =\ new\ Scanner(System.in);}
\item
  \textbf{Scan} the specific type you need: \texttt{nextInt()} or
  \texttt{nextDouble()}.
\end{enumerate}

Happy Coding! 🚀




\end{document}
